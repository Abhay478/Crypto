\documentclass{amsart}
\input{/Users/abhay/Code/Lang/Latex/cheatsheet.tex}
\begin{document}
    \me{Perfect Security, One-Time Pad, etc.}
    \section{Recap}
    \is{Perfect Security}{
        We say that given \(\Pi = \brak{Gen, Enc, Dec}\), \(\forall c \in C\), \(\forall m \in M\), with X as a random variable over M and Y over C, \[\cpr{X = m}{Y = c} = \spr{X = m}\]
    }
    \is{Def 2}{
        \(\forall m_0, m_1 \in M\), \(c \in C\), \[\spr{Enc\brak{m_0, k} = c} = \spr{Enc\brak{m_1, k} = c}\]
    }
    \is{Def 3: Perfect Adversarial Indistinguishability}{
        \begin{itemize}
            \item Adversary picks message randomly from \cbrak{m_0, m_1}, sends to Alice.
            \item Alice encrypts, sends it back.
            \item Adversary guesses.
        \end{itemize}
        
        With \(b \in \cbrak{0, 1}\), \(c = Enc\brak{m_b, k}\), and the adversary guesses \(b' \in \cbrak{0, 1}\), 
        \[\spr{b = b'} = 0.5\]
    }
    \is{One-Time Pad}{
        K = M = C = \(\cbrak{0, 1}^n\)
        \[c = m \oplus k\]
        \is{Issues}{
            \begin{itemize}
                \item len(k) = len(m)
                \item The key cannot be reused.
            \end{itemize}
        }
    }

\section{Problems}
\begin{enumerate}
    \item Prove OTP is PS with Def 2
        To prove: \(\spr{Enc\brak{m_0, k} = c} = \spr{Enc\brak{m_1, k} = c}\)
        where C = M = K = \(\cbrak{0, 1}^n\)
        % LHS = \(\spr{Enc\brak{m_0, k} = c}\) = \(\)
        \begin{align*}
            &\spr{Enc\brak{m_0, k} = c} \\
            &= \spr{m_0 \oplus k = c}\\
            &= \spr{k = c \oplus m_0}\\
            &= \inv{2^n}
        \end{align*}

    \item Prove OTP is PS with Def 3
\end{enumerate}
\end{document}